%%% Socio-Economic Effects of Personal Fabricators
\documentclass[11pt,letterpaper]{article}

\usepackage{tikz}
\usetikzlibrary{mindmap,trees}
\usepackage{verbatim}
\usepackage{researchdiary_png}
\usepackage{listings}
\usepackage{color}

\definecolor{dkgreen}{rgb}{0,0.6,0}
\definecolor{gray}{rgb}{0.5,0.5,0.5}
\definecolor{mauve}{rgb}{0.58,0,0.82}

\lstset{frame=tb,
  language=Python,
  aboveskip=3mm,
  belowskip=3mm,
  showstringspaces=false,
  columns=flexible,
  basicstyle={\small\ttfamily},
  numbers=none,
  numberstyle=\tiny\color{gray},
  keywordstyle=\color{blue},
  commentstyle=\color{dkgreen},
  stringstyle=\color{mauve},
  breaklines=true,
  breakatwhitespace=true
  tabsize=3
}

\newcommand{\workingDate}{\textsc{2013 $|$ July $|$ 25}}
\newcommand{\userName}{Phil Ross \& Wyman Zhao}
\newcommand{\institution}{Binghamton University}

\begin{document}
\univlogo

\title{Research Diary}

{\Huge July 16}\\[5mm]

\textit{These are the research notes on the study of the Socio-Economic Effects of the Widespread Diffusion and \centerline{Adoption of Personal Fabriactors.}}

\section*{Test Cases}

After generating a useable prototype we decided to run simulations testing the following scenarios:

\begin{itemize}
\item
Two Factory Producers
\item
Two Fabricator Producers
\item
One Fabricator \& One Factory Producer
\end{itemize}

Then we decided to test each of those scenarios under the following conditions where producers have:

\begin{itemize}
\item
Similar Priced Goods but Different Goods to Sell
\item
Similar Goods to Sell but Different Prices for Goods
\item
Different Goods to Sell and Different Prices for Goods
\end{itemize}

Based on the equation,

\begin{equation}
d_m = \frac{1}{(t-m)^2} \times \frac{1}{c_m} 
\end{equation}

we ran simulations for 1000 timesteps with the following adjustable parameters:

\begin{itemize}
\item
Number of Goods
\item
Number of Producers
\item
Number of Consumers
\item
Percent of Goods Factory \& Fabricator can Produce Respectively
\item
Factory \& Fabricator Production Rates
\end{itemize}

In addition we used two different methods of choosing which producer to buy from. One involves simply buying from the producer that offers the good for the best closest good-to-cheapest price ratio. The other involves applying weights to each producer to add additional randomness to the buying process.


\end{document}